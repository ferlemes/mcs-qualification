%!TeX root=../tese.tex

%% ------------------------------------------------------------------------- %%
\chapter{Trabalhos relacionados}
\label{cap:trabalhos-relacionados}

Foi realizada uma pesquisa por trabalhos semelhantes e a maioria dos trabalhos
encontrados possui diferenças significantes em relação a proposta desta pesquisa.

A busca foi realizada tanto nas buscas gerais do Google quanto no Google Scholar
com as seguintes palavras-chave:

\begin{itemize}
  \item anomaly detection logs
  \item anomaly detection log records
  \item anomaly detection application
  \item anomaly detection application logs
  \item anomaly detection distributed application
  \item anomaly detection http
  \item anomaly detection load balancer
\end{itemize}

Os trabalhos similares encontrados serão detalhados a seguir.

\section{Anomaly Detection for Application Log Data \citep{grover}}
\label{sec:anomaly-detection-for-application-log-data}

Este trabalho possui uma proposta similar de detectar falhas em uma aplicação em
execução de forma não invasiva, detectando anomalias através da saída de texto
da aplicação.

A extração de características é bem diferente e várias abordagens para detecção
de anomalia são propostas, como: detecção baseada em estatística, agrupamento por
similaridade (``clustering''), distância de uma ocorrência ao grupo similar mais
próximo, aprendizado supervisionado e redes neurais.

Esta pesquisa propõe utilizar a detecção baseada em estatística por acreditar que
é o método mais adequado ao cenário, entretanto estas opções serão melhor estudadas
para serem consideradas na escrita da dissertação.

Em relação à classificação, este trabalho se restringe somente a detecção de uma
anomalia na aplicação sem classificá-la, pois a idéia é somente apontar entre as
milhares de linhas de saída da aplicação o que está fora do padrão.

\section{Anomaly Detection in Log Records \citep{rastogi}}
\label{sec:anomaly-detection-in-log-records}

Este outro trabalho também trata da detecção de anomalia baseada em arquivos de
logs de um servidor web Apache, porém com uma abordagem bem mais simplificada.

A detecção de anomalia proposta é sobre a quantidade de requisições HTTP que
partem de uma mesma origem (endereço IP) ao longo do tempo, e o foco da análise
é de predizer estas ocorrências em um futuro próximo.

Apesar da ideia de detecção ser similar, o propósito do artigo é mais focado em
prever o uso futuro dos servidores web, o que é bem distinto desta pesquisa.

\section{PAD: Performance Anomaly Detection in Multi-server Distributed Systems \citep{pad}}
\label{sec:pad}

Esta publicação descreve um software que foi criado para analisar anomalias de
desempenho em aplicações distribuídas que possuem alta demanda e pedem uma baixa
latência de resposta.

Ela difere bastante por ser mais invasiva e requerer contadores de recursos
(CPU, memória e I/O), assim como contadores de mensagens processadas e número de
processos em uma máquina. Adicionalmente ela utiliza uma quantidade destas
características na casa das centenas, o que é computacionalmente caro.

Também é importante ressaltar que o propósito desta aplicação é de ser utilizada
na execução de um teste de desempenho para análise das anomalias e não monitora
o desempenho de uma aplicação em produção como a que esta pesquisa pretende fazer.

\section{Measuring normality in HTTP traffic for anomaly-based intrusion detection \citep{tapiador}}
\label{sec:measuring-normality-in-http-traffic-for-anomaly-based-intrusion-detection}

Neste trabalho o foco está na detecção de intrusão através de anomalias nas
requisições enviadas a uma aplicação web.

A detecção de anomalia é feita de uma forma bastante diferente utilizando cadeias
de Markov para modelar as sequências regulares que a aplicação recebe, marcando
uma sequência de requisições de baixa probabilidade de ocorrência como uma anomalia.

\section{An Anomaly-Based Approach for Intrusion Detection in Web Traffic \citep{TorranoGimnez2010AnAA}}
\label{sec:an-anomaly-based-approach-for-intrusion-detection-in-web-traffic}

Este outro trabalho propõe a criação de um firewall de requisições a aplicações web,
porém ele requer um treinamento manual que descreve o que deve ser considerado
anomalia ou não.

Apesar do título indicar alguma relação com o que está sendo estudado, ele difere
bastante por não aprender automaticamente o que é anomalia ou não, e pela ação de
bloqueio da requisição diante da anomalia percebida.

\section{Performance Anomaly Detection in Microservice Architectures Under Continuous Change \citep{dullmann}}
\label{sec:performance-anomaly-detection-in-microservice-architectures-under-continuous-change}

Esta dissertação de mestrado é a que mais se aproxima do que esta pesquisa propõe
por abordar o contexto de micro-serviços e se basear na performance da aplicação.

Algumas diferenças podem ser apontadas na etapa de detecção de anomalias, a qual
confia em um limiar pré-determinado para apontar quais requisições são anômalas e
nenhuma tentativa de aprendizado do comportamento da aplicação é utilizada.

O autor menciona problemas interessantes que esta pesquisa deve se deparar, como
atrasos iniciais devido a criação de estruturas de dados, cache ou conexões com
bancos de dados na inicialização de uma instância da aplicação. Este cenário será
verificado durante a pesquisa para avaliar se é possível classificá-lo ou suprimir
alertas gerados por este motivo.

\section{Comparativo}
\label{sec:comparativo}

Na tabela 6.1 serão sumarizadas as diferenças entre os trabalhos relacionados
e esta pesquisa: 

Uma vez identificados alguns trabalhos relacionados, temos algumas ideias de
alternativas para detecção de anomalias e também um cenário novo que deverá ser
levado em conta durante os testes do projeto.

Durante o desenvolvimento do projeto e escrita da dissertação, novas pesquisas
serão realizadas para avaliar o surgimento de outros trabalhos relacionados.

É importante ressaltar que esta pesquisa se torna relevante por não haver
nenhuma outra com a mesma proposta, e também por ser de interesse da indústria
de software, uma vez que tenho apoio da empresa HP Inc. na forma de tempo para
inovação e acesso a dados.

\begin{sidewaystable}
\centering
\caption{Comparativo entre trabalhos relacionados e esta pesquisa.}
\begin{tabular}{lcccccc}
Trabalho relacionado             & Utiliza detecção & Utiliza    & Dispensa       & Realiza       & Realiza o        \\
                                 & de anomalia      & logs da    & instrumentação & classificação & monitoramento    \\
                                 & automática       & aplicação  & do código      & de problemas  & de uma aplicação \\
\hline
6.1\citep{grover}                & \checkmark       & \checkmark & \checkmark     &               & \checkmark       \\
6.2\citep{rastogi}               & \checkmark       & \checkmark & \checkmark     &               & \checkmark       \\
6.3\citep{pad}                   & \checkmark       & \checkmark &                &               &                  \\
6.4\citep{tapiador}              & \checkmark       & \checkmark & \checkmark     &               & \checkmark       \\
6.5\citep{TorranoGimnez2010AnAA} &                  & \checkmark & \checkmark     &               & \checkmark       \\
6.6\citep{dullmann}              &                  & \checkmark & \checkmark     & \checkmark    & \checkmark       \\
\end{tabular}
\end{sidewaystable}
