%!TeX root=../tese.tex

%% ------------------------------------------------------------------------- %%
\chapter{Cronograma}
\label{cap:cronograma}

O experimento relatado no capítulo 5 indica que a pesquisa proposta é viável,
apesar de ser utilizado um cenário bastante simples, onde só foram realizadas
requisições de um único tipo e os registros não tiveram variação no tipo de
resposta recebida.

Para a escrita da dissertação as seguintes atividades serão necessárias:

\begin{enumerate}
  \item Coleta de registros de um serviço real em ambiente de produção, homologação ou qualificação de aplicações web da empresa HP Inc Brasil;
  \item Realização de testes de detecção de anomalias com os dados do serviço real;
  \item Adequação de algoritmos caso necessário;
  \item Realização de testes de classificação das anomalias detectadas;
  \item Implementação de uma aplicação para fazer o monitoramento de registros de um Application Load Balancer da AWS;
  \item Avaliação da aplicação implementada e coleta de resultados.
\end{enumerate}

As atividades elencadas serão executadas nos próximo 12 meses, iniciando
em janeiro de 2020, conforme a tabela abaixo, com o objetivo de concluir
a dissertação ao final de 2020.

\begin{figure}
  \centering
  \begin{ganttchart}{2020-01}{2020-12}
    \gantttitlecalendar{year,month=shortname} \ganttnewline

    \ganttbar[progress=0]{Atividade 1}{2020-01}{2020-01} \ganttnewline
    \ganttbar[progress=0]{Atividade 2}{2020-02}{2020-03} \ganttnewline
    \ganttbar[progress=0]{Atividade 3}{2020-02}{2020-05} \ganttnewline
	\ganttbar[progress=0]{Atividade 4}{2020-04}{2020-05} \ganttnewline
	\ganttbar[progress=0]{Atividade 5}{2020-06}{2020-08} \ganttnewline
	\ganttbar[progress=0]{Atividade 6}{2020-08}{2020-11} \ganttnewline
	\ganttbar[progress=0]{Dissertação}{2020-06}{2020-12} \ganttnewline
    \ganttmilestone{Submissão}{2020-12}
  \end{ganttchart}

  \caption{Cronograma para conclusão da pesquisa.\label{fig:gantt}}
\end{figure}
