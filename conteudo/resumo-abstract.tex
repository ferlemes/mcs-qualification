%!TeX root=../tese.tex

\begin{resumo}{port}
A proposta desta pesquisa é de investigar a detecção de anomalias em uma
aplicação web de maneira automática, classificando-as de forma que ações
corretivas possam ser programadas para manter a aplicação funcionando
adequadamente.

Serão consideradas anomalias as requisições a uma aplicação web que
demoraram mais tempo que o usual, ou que apresentaram erros em suas
respostas. Este tipo de ocorrência pode indicar um problema na aplicação,
na infraestrutura em que ela está sendo executada ou em alguma outra
aplicação da qual esta depende.

O sistema de detecção de anomalias utilizará como dados de entrada os
registros das requisições enviadas a uma aplicação web, os quais podem
ser extraídos de um balanceador de carga ou servidor proxy, sem a
necessidade de instrumentar ou alterar o código da aplicação.

Para a detecção das anomalias será empregado um mecanismo de aprendizado
de máquina autônomo baseado em estatísticas. Para classificação das
anomalias será utilizado um algoritmo de regressão logística que será
pré-treinado com alguns cenários já conhecidos.

A importância deste estudo é de que, em um contexto de micro-serviços,
este tipo de monitoramento é essencial para evitar que problemas em um
componente possam eventualmente comprometer outros e colocar todo o
sistema de software em uma situação degradada ou comprometida.
\end{resumo}
