%!TeX root=../tese.tex

%%%%%%%%%%%%%%%%%%%%%%%%%%%%%%%%%%%%%%%%%%%%%%%%%%%%%%%%%%%%%%%%%%%%%%%%%%%%%%%%
%%%%%%%%%%%%%%%%%%%%%%%%% METADADOS DA QUALIFICAÇÃO %%%%%%%%%%%%%%%%%%%%%%%%%%%%
%%%%%%%%%%%%%%%%%%%%%%%%%%%%%%%%%%%%%%%%%%%%%%%%%%%%%%%%%%%%%%%%%%%%%%%%%%%%%%%%

\mestrado[masc][quali]
\tituloeng{Automatic anomaly detection and classification on web applications}
\orientador[masc]{Prof. Dr. Alfredo Goldman}
\programa{Mestrado em Ciência da Computação}
\programaeng{Master in Computer Science}

\localdefesa{São Paulo}
\datadefesa{7 de janeiro de 2020}
\datadefesaeng{January 7th, 2020}
\ano{2020}
\banca{
  \begin{itemize}
    \item Prof. Dr. Alfredo Goldman - IME-USP [sem ponto final]
	\item Prof. Dr. Daniel Macedo - IME-USP [sem ponto final]
	\item Prof. Dr. Roberto Hirata - IME-USP [sem ponto final]
  \end{itemize}
}

% Palavras-chave separadas por ponto e finalizadas também com ponto.
\palavraschave{Micro-serviços. Monitoramento. Detecção de anomalias. Aprendizado de máquina.}
\keywords{Micro-services. Monitoring. Anomaly detection. Machine learning.}

% Se quiser estabelecer regras diferentes, converse com seu
% orientador
\direitos{Autorizo a reprodução e divulgação total ou parcial
deste trabalho, por qualquer meio convencional ou
eletrônico, para fins de estudo e pesquisa, desde que
citada a fonte.}

% Isto deve ser preparado em conjunto com o bibliotecário
%\fichacatalografica{
% nome do autor, título, etc.
%}

%%%%%%%%%%%%%%%%%%%%%%%%%%% CAPA E FOLHAS DE ROSTO %%%%%%%%%%%%%%%%%%%%%%%%%%%%%

% Embora as páginas iniciais *pareçam* não ter numeração, a numeração existe,
% só não é impressa. O comando \mainmatter (mais abaixo) reinicia a contagem
% de páginas e elas passam a ser impressas. Isso significa que existem duas
% páginas com o número "1": a capa e a página do primeiro capítulo. O pacote
% hyperref não lida bem com essa situação. Assim, vamos desabilitar hyperlinks
% para números de páginas no início do documento e reabilitar mais adiante.
\hypersetup{pageanchor=false}

% A capa; o parâmetro pode ser "port" ou "eng" para definir a língua
\capaime[port]
%\capaime[eng]

% Se você não quiser usar a capa padrão, você pode criar uma outra
% capa manualmente ou em um programa diferente. No segundo caso, é só
% importar a capa como uma página adicional usando o pacote pdfpages.
%\includepdf{./arquivo_da_capa.pdf}

% A página de rosto da versão para depósito (ou seja, a versão final
% antes da defesa) deve ser diferente da página de rosto da versão
% definitiva (ou seja, a versão final após a incorporação das sugestões
% da banca). Os parâmetros podem ser "port/eng" para a língua e
% "provisoria/definitiva" para o tipo de página de rosto.
% Observe que TCCs não têm página de rosto; nesse caso, desabilite
% todas as opções.
%\pagrostoime[port]{definitiva}
\pagrostoime[port]{provisoria}
%\pagrostoime[eng]{definitiva}
%\pagrostoime[eng]{provisoria}

%%%%%%%%%%%%%%%%%%%% DEDICATÓRIA, RESUMO, AGRADECIMENTOS %%%%%%%%%%%%%%%%%%%%%%%

% A definição deste ambiente está no pacote imeusp-capa.sty; se você não
% carregar esse pacote, precisa cuidar desta página manualmente.
%\begin{dedicatoria}
%Esta seção é opcional e fica numa página separada; ela pode ser usada para
%uma dedicatória ou epígrafe.
%\end{dedicatoria}

% Após a capa e as páginas de rosto, começamos a numerar as páginas; com isso,
% podemos também reabilitar links para números de páginas no pacote hyperref.
% Isso porque, embora contagem de páginas aqui começe em 1 e no primeiro
% capítulo também, o fato de uma numeração usar algarismos romanos e a outra
% algarismos arábicos é suficiente para evitar problemas.
\pagenumbering{roman}
\hypersetup{pageanchor=true}

% Agradecimentos:
% Se o candidato não quer fazer agradecimentos, deve simplesmente eliminar
% esta página. A epígrafe, obviamente, é opcional; é possível colocar
% epígrafes em todos os capítulos. O comando "\chapter*" faz esta seção
% não ser incluída no sumário.
%\chapter*{Agradecimentos}
%\epigrafe{Do. Or do not. There is no try.}{Mestre Yoda}
%Texto opcional.
%%!TeX root=../tese.tex
%("dica" para o editor de texto: este arquivo é parte de um documento maior)
% para saber mais: https://tex.stackexchange.com/q/78101/183146

% O resumo é obrigatório, em português e inglês. Este comando também gera
% automaticamente a referência para o próprio documento, conforme as normas
% sugeridas da USP
\begin{resumo}{port}
Elemento obrigatório, constituído de uma sequência de frases concisas e
objetivas, em forma de texto.  Deve apresentar os objetivos, métodos empregados,
resultados e conclusões.  O resumo deve ser redigido em parágrafo único, conter
no máximo 500 palavras e ser seguido dos termos representativos do conteúdo do
trabalho (palavras-chave). Deve ser precedido da referência do documento.
Texto texto texto texto texto texto texto texto texto texto texto texto texto
texto texto texto texto texto texto texto texto texto texto texto texto texto
texto texto texto texto texto texto texto texto texto texto texto texto texto
texto texto texto texto texto texto texto texto texto texto texto texto texto
texto texto texto texto texto texto texto texto texto texto texto texto texto
texto texto texto texto texto texto texto texto.
Texto texto texto texto texto texto texto texto texto texto texto texto texto
texto texto texto texto texto texto texto texto texto texto texto texto texto
texto texto texto texto texto texto texto texto texto texto texto texto texto
texto texto texto texto texto texto texto texto texto texto texto texto texto
texto texto.
\end{resumo}

% O resumo é obrigatório, em português e inglês. Este comando também gera
% automaticamente a referência para o próprio documento, conforme as normas
% sugeridas da USP
\begin{resumo}{eng}
Elemento obrigatório, elaborado com as mesmas características do resumo em
língua portuguesa. De acordo com o Regimento da Pós-Graduação da USP (Artigo
99), deve ser redigido em inglês para fins de divulgação. É uma boa ideia usar
o sítio \url{www.grammarly.com} na preparação de textos em inglês.
Text text text text text text text text text text text text text text text text
text text text text text text text text text text text text text text text text
text text text text text text text text text text text text text text text text
text text text text text text text text text text text text.
Text text text text text text text text text text text text text text text text
text text text text text text text text text text text text text text text text
text text text.
\end{resumo}

%!TeX root=../tese.tex

\begin{resumo}{port}
A proposta desta pesquisa é de investigar a detecção de anomalias em uma
aplicação web de maneira automática, classificando-as de forma que ações
corretivas possam ser programadas para manter a aplicação funcionando
adequadamente.

Serão consideradas anomalias as requisições a uma aplicação web que
demoraram mais tempo que o usual, ou que apresentaram erros em suas
respostas. Este tipo de ocorrência pode indicar um problema na aplicação,
na infraestrutura em que ela está sendo executada ou em alguma outra
aplicação da qual esta depende.

O sistema de detecção de anomalias utilizará como dados de entrada os
registros das requisições enviadas a uma aplicação web, os quais podem
ser extraídos de um balanceador de carga ou servidor proxy, sem a
necessidade de instrumentar ou alterar o código da aplicação.

Para a detecção das anomalias será empregado um mecanismo de aprendizado
de máquina autônomo baseado em estatísticas. Para classificação das
anomalias será utilizado um algoritmo de regressão logística que será
pré-treinado com alguns cenários já conhecidos.

A importância deste estudo é de que, em um contexto de micro-serviços,
este tipo de monitoramento é essencial para evitar que problemas em um
componente possam eventualmente comprometer outros e colocar todo o
sistema de software em uma situação degradada ou comprometida.
\end{resumo}



%%%%%%%%%%%%%%%%%%%%%%%%%%% LISTAS DE FIGURAS ETC. %%%%%%%%%%%%%%%%%%%%%%%%%%%%%

% Como as listas que se seguem podem não incluir uma quebra de página
% obrigatória, inserimos uma quebra manualmente aqui.
\makeatletter
\if@openright\cleardoublepage\else\clearpage\fi
\makeatother

% Todas as listas são opcionais; Usando "\chapter*" elas não são incluídas
% no sumário. As listas geradas automaticamente também não são incluídas
% por conta das opções "notlot" e "notlof" que usamos mais acima.

% Normalmente, "\chapter*" faz o novo capítulo iniciar em uma nova página, e as
% listas geradas automaticamente também por padrão ficam em páginas separadas.
% Como cada uma destas listas é muito curta, não faz muito sentido fazer isso
% aqui, então usamos este comando para desabilitar essas quebras de página.
% Se você preferir, comente as linhas com esse comando e des-comente as linhas
% sem ele para criar as listas em páginas separadas. Observe que você também
% pode inserir quebras de página manualmente (com \clearpage, veja o exemplo
% mais abaixo).
\newcommand\disablenewpage[1]{{\let\clearpage\par\let\cleardoublepage\par #1}}

% Nestas listas, é melhor usar "raggedbottom" (veja basics.tex). Colocamos
% a opção correspondente e as listas dentro de um par de chaves para ativar
% raggedbottom apenas temporariamente.
{
\raggedbottom

%%%%% Listas criadas manualmente

%\chapter*{Lista de Abreviaturas}
\disablenewpage{\chapter*{Lista de Abreviaturas}}

\begin{tabular}{rl}
	ALB          & Balanceador de carga de aplicação da AWS (\emph{Application Load Balancer}) \\
	AWS			 & Amazon Web Services \\
	CPU          & Unidade de Processamento Central (\emph{Central Processing Unit}) \\
	EC2          & Nuvem Computacional Elástica da AWS (\emph{Elastic Compute Cloud}) \\
    HTTP         & Protocolo de Transferência de Hipertexto (\emph{HyperText Transfer Protocol}) \\
    HTTPS        & Protocolo Seguro de Transferência de Hipertexto (\emph{HyperText Transfer Protocol Secure}) \\
	I/O          & Entrada e Saída (de Dados) (\emph{Input and Output}) \\
	RAM          & Memória de Acesso Randômico (\emph{Random Access Memory}) \\
	S3           & Sistema Simples de Armazenamento da AWS (\emph{Simple Storage Service}) \\
	SOA          & Arquitetura Orientada a Serviços (\emph{Service Oriented Architecture}) \\
	URL          & Localizador Uniforme de Recursos (\emph{Uniform Resource Locator}) \\
	vCPU         & Unidade de Processamento Central Virtual (\emph{Virtual CPU})
\end{tabular}

% Quebra de página manual
\clearpage

%%%%% Listas criadas automaticamente

%\listoffigures
\disablenewpage{\listoffigures}

%\listoftables
\disablenewpage{\listoftables}

% Esta lista é criada "automaticamente" pela package float quando
% definimos o novo tipo de float "program" (em utils.tex)
%\listof{program}{\programlistname}
\disablenewpage{\listof{program}{\programlistname}}

% Sumário (obrigatório)
\tableofcontents

} % Final de "raggedbottom"
