%%%%%%%%%%%%%%%%%%%%%%%%%%%%%%%%%%%%%%%%%%%%%%%%%%%%%%%%%%%%%%%%%%%%%%%%%%%%%%%%
%%%%%%%%%%%%%%%%%%%%%%% SUMÁRIO, CABEÇALHOS, SEÇÕES %%%%%%%%%%%%%%%%%%%%%%%%%%%%
%%%%%%%%%%%%%%%%%%%%%%%%%%%%%%%%%%%%%%%%%%%%%%%%%%%%%%%%%%%%%%%%%%%%%%%%%%%%%%%%

% Formatação personalizada do sumário, lista de tabelas/figuras etc.
%\usepackage{titletoc}

% Coloca as linhas "Apêndices" e "Anexos" no sumário. Com a opção "inline",
% cada apêndice/anexo aparece como "Apêndice X" ao invés de apenas "X".
\dowithsubdir{extras/}{\usepackage{appendixlabel}}

% titlesec permite definir formatação personalizada de títulos, seções etc.
% Observe que titlesec é incompatível com os comandos refsection
% e refsegment do pacote biblatex!
\makeatletter
\ltx@IfUndefined{chapter}
    {
        % A classe atual não define "chapter" e, portanto, não faz sentido
        % carregar imagechapter. Ao invés disso, vamos usar titlesec apenas
        % para fazer títulos, seções etc. não serem justificados.
        \usepackage[raggedright]{titlesec}
    }
    {
        % Esta package utiliza titlesec e implementa a possibilidade de incluir
        % uma imagem no título dos capítulos com o comando \imgchapter (leia
        % os comentários no arquivo da package).
        \dowithsubdir{extras/}{\usepackage{imagechapter}}
    }
\makeatother

% Permite saber o número total de páginas; útil para colocar no
% rodapé algo como "página 3 de 10" com "\thepage de \pageref{LastPage}"
%\usepackage{lastpage}

% Permite definir cabeçalhos e rodapés
%\usepackage{fancyhdr}

% Personalização de cabeçalhos e rodapés com o estilo deste modelo
\dowithsubdir{extras/}{\usepackage{imeusp-headers}}

% Só olha até o nível 2 (seções), ou seja, não coloca nomes de
% subseções ou subsubseções no sumário (nem nos cabeçalhos).
\setcounter{tocdepth}{2}


%%%%%%%%%%%%%%%%%%%%%%%%%%%%%%%%%%%%%%%%%%%%%%%%%%%%%%%%%%%%%%%%%%%%%%%%%%%%%%%%
%%%%%%%%%%%%%%%%%%%%%%%%%% ESPAÇAMENTO E ALINHAMENTO %%%%%%%%%%%%%%%%%%%%%%%%%%%
%%%%%%%%%%%%%%%%%%%%%%%%%%%%%%%%%%%%%%%%%%%%%%%%%%%%%%%%%%%%%%%%%%%%%%%%%%%%%%%%

% LaTeX por default segue o estilo americano e não faz a indentação da
% primeira linha do primeiro parágrafo de uma seção; este pacote ativa essa
% indentação, como é o estilo brasileiro
\usepackage{indentfirst}

% A primeira linha de cada parágrafo costuma ter um pequeno recuo para
% tornar mais fácil visualizar onde cada parágrafo começa. Além disso, é
% possível colocar um espaço em branco entre um parágrafo e outro. Esta
% package coloca o espaço em branco e desabilita o recuo; como queremos
% o espaço *e* o recuo, é preciso guardar o valor padrão do recuo e
% redefini-lo depois de carregar a package.
\newlength\oldparindent
\setlength\oldparindent\parindent
\usepackage[parfill]{parskip}
\setlength\parindent\oldparindent
